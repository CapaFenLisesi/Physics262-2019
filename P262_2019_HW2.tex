\documentclass[12pt]{article}
%% arXiv paper template by Flip Tanedo
%% last updated: Dec 2016


%%%%%%%%%%%%%%%%%%%%%%%%%%%%%
%%%  THE USUAL PACKAGES  %%%%
%%%%%%%%%%%%%%%%%%%%%%%%%%%%%

\usepackage{amsmath}
\usepackage{amssymb}
\usepackage{amsfonts}
\usepackage{graphicx}
\usepackage{xcolor}
\usepackage{nopageno}
\usepackage{enumerate}
\usepackage{parskip}

%%%%%%%%%%%%%%%%%%%%%%%%%%%%%%%%%
%%%  UNUSUAL PACKAGES        %%%%
%%%  Uncomment as necessary. %%%%
%%%%%%%%%%%%%%%%%%%%%%%%%%%%%%%%%

%% MATH AND PHYSICS SYMBOLS
%% ------------------------
%\usepackage{slashed}       % \slashed{k}
%\usepackage{mathrsfs}      % Weinberg-esque letters
%\usepackage{youngtab}	    % Young Tableaux
%\usepackage{pifont}        % check marks
\usepackage{bbm}           % \mathbbm{1} incomp. w/ XeLaTeX
%\usepackage[normalem]{ulem} % for \sout


%% CONTENT FORMAT AND DESIGN (below for general formatting)
%% --------------------------------------------------------
\usepackage{lipsum}        % block of text (formatting test)
%\usepackage{color}         % \color{...}, colored text
%\usepackage{framed}        % boxed remarks
%\usepackage{subcaption}    % subfigures; subfig depreciated
%\usepackage{paralist}      % compactitem
%\usepackage{appendix}      % subappendices
%\usepackage{cite}          % group cites (conflict: collref)
%\usepackage{tocloft}       % Table of Contents

%% TABLES IN LaTeX
%% ---------------
%\usepackage{booktabs}      % professional tables
%\usepackage{nicefrac}      % fractions in tables,
%\usepackage{multirow}      % multirow elements in a table
%\usepackage{arydshln} 	    % dashed lines in arrays

%% Other Packages and Notes
%% ------------------------
%\usepackage[font=small]{caption} % caption font is small



\renewcommand{\thesection}{}
\renewcommand{\thesubsection}{\arabic{subsection}}

%%%%%%%%%%%%%%%%%%%%%%%%%%%%%%%%%%%%%%%%%%%%%%%
%%%  PAGE FORMATTING and (RE)NEW COMMANDS  %%%%
%%%%%%%%%%%%%%%%%%%%%%%%%%%%%%%%%%%%%%%%%%%%%%%

\usepackage[margin=2cm]{geometry}   % reasonable margins

\graphicspath{{figures/}}	        % set directory for figures

% for capitalized things
\newcommand{\acro}[1]{\textsc{\MakeLowercase{#1}}}

\numberwithin{equation}{subsection}    % set equation numbering
\renewcommand{\tilde}{\widetilde}   % tilde over characters
\renewcommand{\vec}[1]{\mathbf{#1}} % vectors are boldface

\newcommand{\dbar}{d\mkern-6mu\mathchar'26}    % for d/2pi
\newcommand{\ket}[1]{\left|#1\right\rangle}    % <#1|
\newcommand{\bra}[1]{\left\langle#1\right|}    % |#1>
\newcommand{\Xmark}{\text{\sffamily X}}        % cross out


\let\olditemize\itemize
\renewcommand{\itemize}{
  \olditemize
  \setlength{\itemsep}{1pt}
  \setlength{\parskip}{0pt}
  \setlength{\parsep}{0pt}
}


% Commands for temporary comments
\newcommand{\comment}[2]{\textcolor{red}{[\textbf{#1} #2]}}
\newcommand{\flip}[1]{{\color{red} [\textbf{Flip}: {#1}]}}
\newcommand{\email}[1]{\texttt{\href{mailto:#1}{#1}}}

\newenvironment{institutions}[1][2em]{\begin{list}{}{\setlength\leftmargin{#1}\setlength\rightmargin{#1}}\item[]}{\end{list}}


\usepackage{fancyhdr}		% to put preprint number



%%%%%%%%%%%%%%%%%%%
%%%  HYPERREF  %%%%
%%%%%%%%%%%%%%%%%%%

%% This package has to be at the end; can lead to conflicts
\usepackage{microtype}
\usepackage[
	colorlinks=true,
	citecolor=black,
	linkcolor=black,
	urlcolor=green!50!black,
	hypertexnames=false]{hyperref}



%%%%%%%%%%%%%%%%%%%%%
%%%  TITLE DATA  %%%%
%%%%%%%%%%%%%%%%%%%%%

\begin{document}


\begin{center}

    {\Large \textsc{Homework 2:}
    \textbf{Tensor Stuff}}

\end{center}

\vskip .4cm

\noindent
\begin{tabular*}{\textwidth}{rl}
	\textsc{Course:}& Physics 262, \emph{Group Theory for Physicists} (Fall 2019)
	\\
	\textsc{Instructor:}& Professor Flip Tanedo (\email{flip.tanedo@ucr.edu})
	\\
	\textsc{Due by:}& Be ready to discuss on Mon, Feb 25
	% \\
	% \textsc{Updated:}& Jan 24, 11:30am

	%
\end{tabular*}



\subsection{Weight Diagram Example}

Draw the SU(3) weight diagram for the SU(3) representation with highest weight $(3/2, \, \sqrt{3}/2)$. What is the dimension of this representation? It may be helpful to use the triangular graph paper available on the course website.


\subsection{Tensor Representation}

What is the decomposition of $\mathbf{3}\otimes\mathbf{3}\otimes\mathbf{3}$ into SU(3) irreducible representations? Is there a combination that is invariant?

\subsection{Interpretation: SU(3) Color Symmetry}

\subsubsection{The $\Delta^{++}$}

The $\Delta^{++}$ is a composite fermion that is a cousin of the proton. One puzzle from the early days of modern particle physics was why the $\Delta^{++}$ existed. We suspected that the $\Delta^{++}$ was composed of three identical fermions, each with charge 2/3 (we now know this as the up quark). However, identical fermion wavefunctions must be antisymmetrized. Thus the $\Delta^{++}$ could not be made up of identical fermions. Explain why this was indirect motivation for SU(3) color symmetry for which the up quark is in the fundamental representation. \textsc{Hint}: your explanation should make use of a SU(3) invariant tensor.

\subsubsection{Not SU(2)}

Argue that in order to account for the $\Delta^{++}$, the new symmetry group could not have been SU(2). There are many ways to argue this, but for this problem, argue based on tensor representations of SU(2).

\subsubsection{Not SU(4)}

Argue that in order to account for the $\Delta^{++}$, the new symmetry group could not have been SU(4). There are many ways to argue this, but for this problem, argue based on invariant tensors.


\subsubsection{``Technigluons''}

Suppose I have a new theory based on SU($N$) symmetry. I have the following particles:
\begin{enumerate}
  \item A techni-quark in the fundamental representation.
  \item A techni-tensor with $(N-1)$ fundamental (upper) indices.
  \item A techni-gluon in the adjoint representation.
\end{enumerate}
Particles are allowed to interact if I can write down an invariant combination of their tensor product. Argue that the three particles above may interact with one another.

\subsection{Adjoint vs. Fundamental--Anti-fundamental}

We saw that the tensor product of a fundamental and an anti-fundamental of SU(2) decomposes as $$\mathbf{2}\otimes\bar{\mathbf{2}} = \mathbf{3}\oplus\mathbf{1}.$$
One way to understand the $\mathbf{3}$ on the right-hand side is to see how it transforms as a spin-1 representation versus as a tensor product.

\subsubsection{Tensor Product Transformation}

Recall that SU($N$) indices transform as follows:
\begin{align}
  \psi^i &\to U^i_{\phantom{i}j} \psi ^j
  &
  \chi_i &\to  \chi_j (U^\dag)^j_{\phantom{j}i} \ ,
\end{align}
where $U$ is a finite SU($N$) transformation.

\begin{enumerate}
  \item Write the infinitesimal transformation of $\Psi^i_{\phantom{i}j}$ under an arbitrary, small SU(2) transformation.
  \item Let $\vec{v} = (v^+, v^0, v^-)$ be a state in the $\mathbf{3}$ of the aforementioned tensor product. With respect to the fundamental and the antifundamental indices, $\mathbf{v}$ may be written
  \begin{align}
    \vec{v} = v^+ (T^+)^i_{\phantom{i}j} + v^0 (T^3)^i_{\phantom{i}j} + v^- (T^-)^i_{\phantom{i}j} \ .
  \end{align}.
  Write the components of $\vec{v}$ after performing an infinitesimal SU(2) transformation in the $T^1$ direction according to the transformation rule you wrote above.
\end{enumerate}

\subsubsection{Triplet Transformation}
Alternatively, we may think of $\vec{v}$ as an object with only a triplet index:
\begin{align}
  \vec{v} = v^+ \ket{m=1} + v^0 \ket{m=0} + v^- \ket{m=-1} \ .
\end{align}
\begin{enumerate}
  \item Write the infinitesimal transformation of $v^a$ under an arbitrary, small SU(2) transformation, where $a$ is a triplet (spin-1) index.
  \item Write the components of $\vec{v}$ after performing an infinitesimal SU(2) transformation in the $T^1$ direction according to the transformation rule you wrote above.
\end{enumerate}

\textsc{Discussion}: In this way, the tensor $T$ is used to convert between different kinds of indices. Observe that the transformation on the components in $\vec{v}$ are the same whichever picture you use.


\subsection{Invariance of $\varepsilon^{i_1\cdots i_N}$}

Show that $\varepsilon^{i_1\cdots i_N}$ is invariant under an SU$(N)$ transformation. \textsc{Hint}: you can check for SU(2) explicitly. Use a handy expression for the determinant of an $N\times N$ matrix.


\section*{Extra Credit}

These problems are for your own edification. You are encouraged to explore them according to your own personal and research interests.
%
%

\subsection*{Normalizations once and for all}

In case you'd like to keep track of normalization, write a set of notes (for yourself) that go from first principles to the normalizations that we use in our general study of roots and weights.  Start with Georgi section 2.4 which starts with commutation relations and leads up to (2.37):
\begin{align}
  \text{Tr}(T^aT^b) = \lambda\delta^{ab} \ .
\end{align}
Then jump to section 6.3 and the inner product defined on the adjoint representation:
\begin{align}
  \langle E_\alpha | E_\beta \rangle
  &= \frac{1}{\lambda}\text{Tr}\left(E^\dag_\alpha E_\beta \right) = \delta_{\alpha\beta} \ .
\end{align}
Confirm that this also gives $\langle H_i | H_j \rangle = \delta_{ij}$ on the Cartan subalgebra. Confirm that with this convention,
\begin{align}
  [E_\alpha, E_{-\alpha}] = \alpha \cdot H \ .
\end{align}
\textsc{Hint}: We know that $[E_\alpha, E_{-\alpha}]$ is some linear combination of the Cartan elements because $E_\alpha|E_{-\alpha}\rangle$ has weight zero.

Finally, show that the SU(2) subalgebra along the $\alpha$ direction is generated by
\begin{align}
  E^3 &= \frac{\alpha\cdot H}{\alpha^2}
  &
  E^\pm &= \frac{E_{\pm \alpha}}{|\alpha|} \ .
\end{align}
Note that $[E^+, E^-] = E^3$. This is probably indicate that the normalization that we've been using in this course has been the `wrong' one.


\end{document}
